\documentclass[a4paper, 12pt]{article}

\usepackage[utf8]{inputenc} % for UTF-8 support

\title{AI for IMGD Final Project Report}
\author{Cole Granof \and Joseph Petitti \and Matthew Puentes}
\date{\today}

\begin{document}

\maketitle

\section{What We Did}

For the final AI for IMGD project we created a procedurally generated, top-down,
twin stick shooter game, inspired by games like \textit{The Binding of Isaac}
and \textit{Nuclear Throne}. In the game, the player controls a small
eyeball-like character that can move around the game world, a large series of
caves. There are several types of enemies that inhabit the caves, including some
that chase the player, others that wander around randomly, and even some that
can shoot bullets. Players fight these enemies while exploring the cave to track
down three Boss enemies, which must be defeated to win. They can also find power
ups hidden the caves, powerful artifacts that increase the player's abilities in
unique ways.

The hero is controlled with a keyboard or gamepad, and can move around and shoot
in any direction. Like all game entities in our custom engine, the hero has a
velocity, acceleration, and drag. This allows it realistically reach a maximum
speed in a way that is simple to program and maintain.

There are three main enemy types: Chase, Scatter, and Shooter. Chase enemies
sleep until the player comes close or shoots at them, but chase after the hero
as soon as they wake up. Scatter enemies wander around the board in random and
unpredictable ways. Shooter enemies maintain their distance and fire slow-moving
projectiles at the hero. Each enemy has a random chance to be bigger. Bigger
enemies split into several smaller enemies when they are killed, providing
varying levels of difficulty.

Each enemy type has a different type of face, which is further modified by
randomly generated parameters. For example, Chase enemies have one large eye in
the center of their face, but the size and positioning of their eye and mouth is
tweaked by parameters. The game world has randomly generated color scheme, which
is applied to the cave environment as well as enemies.

The game also features a robust power up system. All power ups are applicable to
both the hero and the enemies---both are abstracted to a Creature superclass.
All Creatures share functionality such as speed, shot rate, and bullet damage.
In this way, it's easy to create unique and challenging enemies by applying
randomized power ups to them. We plan to eventually have 26 different power
ups (one for each letter of the alphabet), but for this early beta version we
have only seven.

To make our game accessible to as many people as possible we implemented it in
JavaScript using the HTML5 canvas API for graphics. To give ourselves the most
control over the game environment we decided to write absolutely everything
ourselves, from scratch. We use no game engine, no third party libraries, no
preprocessor, no front-end build tools, and no graphics API but the one built in
to every browser. We implemented a full-fledged game engine in JavaScript that
runs game logic independent of screen refresh rate, handles collisions in an
efficient way, has a robust collision-mapping system using first-class
functions, implements basic kinematics and particles, has customizable camera
controls, and uses generalized keyboard and gamepad input.

\section{Why We Did It}

% TODO write

\section{How It Relates to Things We Read}

% TODO write

\section{How It Works}

% TODO write


\section{Evaluation Results}

% TODO write


\section{What Those Results Mean}

% TODO write


\section{What We Learned}

% TODO write

\end{document}

