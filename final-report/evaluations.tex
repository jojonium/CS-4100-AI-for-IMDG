\documentclass[a4paper, 12pt]{article}

\usepackage[utf8]{inputenc} % for UTF-8 support

\title{AI for IMGD Final Project Evaluations}
\author{Cole Granof \and Joe Petitti \and Matt Puentes}
\date{\today}

\begin{document}

\maketitle

\section{Matt Puentes}

I worked half/half on implementing engine features and game-specific features.
My main contribution was on Collision, which I built from scratch. The collision
works fairly well, with every entity having a-b rectangle collision with any
other entities. The collision interacts well between entities and the world as
well, allowing smooth movement and efficient terrain detection. It was took a
while to get all of the implementation details right and to do it efficiently, 
but it meant that later on we were able to fit a lot of entities on screen
without a slowdown.

My other big contribution was on the level generation. I created both the
flood fill and cave connection systems described in the sections above, helping
the game world feel more like we intended.

Additionally, I worked with Joe, Cole and our play testers to help balance the
game (capping power-up multipliers, changing world generation parameters, etc.)

\section{Cole Granof}

I worked on implementing game engine features so that my teammates would be able
to easily add functionality to our game. This included a particle system, basic
kinematics, and a system from adding behaviors to entities in the world. This
was a lot of work to do without any external libraries, but I am happy with the
result.

Throughout the development, I added various optimizations so that we could have
larger worlds with more enemies, all with minimal to no slowdown.

I took the first pass at generating random cave levels using cellular automata.
Matt improved on this by connecting the disconnected caves with breakable walls
so that the entire cave was traversable.

\end{document}

