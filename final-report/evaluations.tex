\documentclass[a4paper, 12pt]{article}

\usepackage[utf8]{inputenc} % for UTF-8 support

\title{AI for IMGD Final Project Evaluations}
\author{Cole Granof \and Joe Petitti \and Matt Puentes}
\date{\today}

\begin{document}

\maketitle

\section{Cole Granof}

I worked on implementing game engine features so that my teammates would be able
to easily add functionality to our game. This included a particle system, basic
kinematics, and a system from adding behaviors to entities in the world. This
was a lot of work to do without any external libraries, but I am happy with the
result.

Throughout the development, I added various optimizations so that we could have
larger worlds with more enemies, all with minimal to no slowdown.

I took the first pass at generating random cave levels using cellular automata.
Matt improved on this by connecting the disconnected caves with breakable walls
so that the entire cave was traversable.

Brainstorming and coordinating programming tasks went well during development. I
was very happy with what everyone brought to the table and our workflow with
GitHub. However, we might have over-scoped for this assignment, but this project
is something we want to continue work on, so this is a good thing in many ways.
For all the effort I put in, I believe I deserve an A.

\section{Joseph Petitti}

I worked mostly on the power-up system, implementing the Creature superclass and
setting up the system to allow for general-purpose power-ups that can apply to
both the player character and the enemies. For many of the commonly used
actions, such as shooting, a Creature has an array of functions that are
executed when that action takes place. Because functions are first-class
citizens in JavaScript, they can be passed as arguments or returned from
other functions, making this behavior much easier. Because of the Creature
abstraction, power-ups can be applied to enemies the same way they are applied
to the hero, allowing us to use randomly generated power-ups to create stronger,
more interesting enemies.

I think our team worked pretty well together, since we are all friends and have
worked together on projects in the past. We ended up creating a really cool game
that we plan to work on more in the future.

Perhaps we could have done a better job of splitting up the work. Early on in
the project Matt and I were swamped with work from other classes and Cole ended
up implementing a lot of the early engine features on his own. In future
projects I will likely set up explicit deadlines for each team member to prevent
this from happening again.

Each member of the team put in a ton of effort for this project---probably
dozens of hours each. We wrote efficient, well-organized, and well-documented
code, to such a degree that we all want to continue working on the project
afterward. Further, I learned a lot about procedural generation and
parameterized design from completing this project, which will influence projects
I do in the future. For theses reasons, I think every member of our group
deserves an A.

\section{Matt Puentes}

I worked half/half on implementing engine features and game-specific features.
My main contribution was on Collision, which I built from scratch. The collision
works fairly well, with every entity having a-b rectangle collision with any
other entities. The collision interacts well between entities and the world as
well, allowing smooth movement and efficient terrain detection. It was took a
while to get all of the implementation details right and to do it efficiently, 
but it meant that later on we were able to fit a lot of entities on screen
without a slowdown.

My other big contribution was on the level generation. I created both the
flood fill and cave connection systems described in the sections above, helping
the game world feel more like we intended.

Additionally, I worked with Joe, Cole and our play testers to help balance the
game (capping power-up multipliers, changing world generation parameters, etc.)

I feel as if all of my implementations went well, but if I could do it again, I
would have spent more time on collision to get circle/rectangle collision, as we
ended up having a large amount of bigger circular enemies, so sometimes their
hitbox corners collide when they don't need to. This isn't super noticeable or
game-breaking, but I would like to have fixed it before the project was over.

I think I (and all my partners) deserve an A on this assignment. I learned quite
a bit about 2D generative coding, and the amount we got done while practicing
good coding techniques and version control management on top of our busy
schedules was as testament to our teamwork and goal setting.

\end{document}

