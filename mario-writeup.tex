\documentclass[a4paper, 10pt]{article}

\usepackage[utf8]{inputenc} % UTF-8 support
\usepackage{tikz} % for diagrams
\usepackage{forest}

\title{Mario Agent Turing Test Writeup}
\author{Grant Ferguson and Joseph Petitti}
\date{\today}

\begin{document}
\maketitle

\section{Design and Behavior Goals}

Our overall design and behavior goal was to create an AI agent that can navigate
Mario levels successfully while also behaving like a human being to the best of
its ability. More specifically, we decided that in pursuit of these goals the
more detail oriented goals would be:

\begin{enumerate}
	\item To have the AI agent exhibit traits that many of players demonstrated
	\item To have the agent exhibit distinctly human perceived thoughts (i.e.
		stopping to think before acting)
	\item To have the agent be able to do seemingly random movements for a
		non-perceivable reason, like a fallible human player
\end{enumerate}

We chose these specific goals for a few reasons, the first being that these were
tasks that we believed would lead to the most human like agent possible, while
also being able to be built using the framework. Another key reason that we
thought that these would be proper goals to have for our agent was that with
these goals, most of the user testing that we did would have a similar trait or
traits to these rules. By the end of the project, most of these specific goals
we feel were met, and in doing so the agent was somewhat human-like in its
behaviors.

\section{Implementation}

We implemented a decision tree to drive our AI. We created a node interface
(\texttt{INode}) that requires an \texttt{execute()} method. There are two types
of nodes that implement \texttt{INode}, decision nodes and action nodes.
Decision nodes override \texttt{execute()} to execute one of their various
\texttt{INode} branches based on some condition, and action nodes override
\texttt{execute()} to return a specific action. This way we can recursively
iterate down the decision tree by \texttt{execute()} on the root decision node.
We implemented various decision nodes based on conditions such as an enemy being
nearby, Mario standing on a ledge, or simply randomly selecting a child branch.
We also made several action nodes, such as walking left, jumping right, or doing
nothing. The complete decision tree is shown in Figure~1.

\begin{figure}[h]
	\centering
	\begin{forest}
		[LevelJustStarted?,draw
			[DoNothing.,ellipse,draw]
			[InAir?,draw
				[GoForward.,ellipse,draw]
				[EnemyVeryClose?,draw
					[GoLeft.,ellipse,draw]
					[EnemyNearby?,draw
						[GoForward.,ellipse,draw]
						[OnLedge?,draw
							[50-50?,draw
								[GoLeft.,ellipse,draw]
								[JumpRight.,ellipse,draw]
							]
							[GoForward,ellipse,draw]
						]
					]
				]
			]
		]
	\end{forest}
	\label{fig:decisionTree}
	\caption{Left branches represent ``yes'', right branches represent ``no''}
\end{figure}

\section{Testing Process}

For testing our agent to see what the actual results were of each of the nodes
(questions, actions, etc.) we decided that the best approach would be to run
through actual original Mario levels and see particular edge cases where either
the agent got stuck or the agent had some interactions that we did not
previously account for in the decision tree itself. In doing so we would
hopefully discover situations that would help us identify places to add
additional nodes into the tree where the tree itself was lacking proper
interactions with certain level elements built into certain levels. Another
aspect of testing the agent has to do with how the agent was initially created.
Since we used the A* methods from the \texttt{robinBaumgarten} agent inside of
the Mario framework, part of the testing was seeing the limitations of that
system and to what extent we could use it to achieve parts of the tree and other
actions. Upon testing through those classes, we discovered that there was not a
way to move to a specific location in the level, and that that would have to be
done manually.

\section{Pros and Cons}

The pros and cons of a decision tree are fairly apparent once it is implemented
in any agent, and though there are situations where they can be mitigated, they
are still interesting and important to look at when deciding on what type of
system to use for an AI agent. To start, the benefits of using a decision tree
are fairly straightforward when using the Mario Framework, especially if using
it for the agent setup inside a package. When you use a decision tree, it is
fairly simple to build ones the pieces are set up, and it is also fairly easy to
explain to other users or developers of the system. Another advantage to using
decisions trees is the low processing time for each decision, as each decision
node is a singular boolean expression in code, and getting to an action requires
at most the amount of levels in the tree expressions. So in a binary decision
tree, which would have the minimum amount of nodes per layer, you could have 16
possible actions in 4 layers, and that is with only using binary layers.

Some disadvantages are also apparent when using a decision tree over other
methods in the Mario framework. One of these disadvantages is that when you use
a decision tree, it can be limited in the framework presented based on the
amount of decisions you can make in a frame to frame basis based off of no
previously knowledge of the level compared to other systems you can use to make
the agent seem more human. Decision trees can't store a state, so it's difficult
to make complex actions that rely on the previous actions.

Some things that we would change if we were to do the project again would be to
try and find  a way to get specific input locations from the A* methods to be
able to move Mario with more precision when making decisions, as well as being
able to show that when Mario moves there would be a specific function to go to a
spot using A* instead of in the general forward direction. It may have been a
poor decision to use a decision tree in the first place, and a more flexible
method such as a utility table may have been better suited for this task.

\end{document}
