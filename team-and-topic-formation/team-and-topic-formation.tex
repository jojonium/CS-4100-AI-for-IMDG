\documentclass[a4paper, 10pt, american]{article}

% useful packages
\usepackage[utf8]{inputenc} % UTF-8 support
\usepackage[american]{babel} % for changing particular titles
%\usepackage[margin=1in]{geometry} % set 1in margins
\usepackage{hyperref} % hyperlinks
\usepackage{natbib} % for bibliography

\title{Term Project: Team and Topic Formation}
\author{Cole~Granof, Joseph~Petitti, Matthew~Puentes}
\date{\today}

\begin{document}

\maketitle

\begin{itemize}
	\item Team members: Cole Granof, Joseph Petitti, and Matthew Puentes
	\item Topic for our project: ``procedurally generated levels and enemies for
		a top-down roguelite twinstick shooter.''
	\item We hope to create an engaging and unique game, similar to other
		procedurally generated games like \textit{The Binding of Isaac} and
		\textit{Nuclear Throne}, using the principles of parameterized design we
		explored in this class. Levels will be generated to form natural-looking
		cave-like structures, enemies will be randomized by assigning them
		points in various attributes and giving them random facial
		characteristics, and power-ups will also be randomly generated. In this
		way the game will be different each time it is played.
	\item This project is interesting to us because not many existing roguelites
		utilize both procedurally generated levels and procedurally generated
		enemies. The combination of these two techniques can allow for a wide
		variety of different encounters and unique gameplay scenarios. We
		believe that cellular automata can be used to create really cool levels,
		and this technique is underused in existing games.
	\item We just want to make a fun game that other people can enjoy. Hopefully
		the procedural nature of the levels and enemies will give the game
		longevity. There isn't a high potential for generating offensive content
		based on the constraints we use on the engine.
	\item Three external sources we have read:
		\begin{itemize}
			\item \cite{liu2017}. The researchers of this paper used an AI agent
				to iteratively adjust the difficulty of a game based on the AI
				agent's performance. The researchers adjusted various parameters
				that dictate different aspects of difficulty in order to create
				a balanced experience. Essentially, this paper provides an
				outline for automating play testing, as well as a breakdown of
				gameplay elements that increase difficulty when individually
				scaled.

				Performing this kind of AI-powered play testing is likely
				outside the scope of our project. However, this paper separates
				the difficulty of their ``pace-battle game'' into various
				parameters, such as speed of the player, the speed of enemies
				and the amount of bullets fired. Breaking down difficulty into
				these discrete components will help us generate enemies with
				specific levels of difficulty. Furthermore, with enough
				well-chosen parameters, we can create enemies that challenge the
				player in different ways. This will hopefully encourage players
				to change up their strategies.

			\item \cite{pell1992}. This paper details a way to fully specify and
				generate chess-like games such as Western Chess, Xiangqi, Shogi
				and checkers. Capture rules and piece movement can be specified
				to encompass any existing chess like games, and, of course,
				define completely new games as well.

				This inspired the idea of randomly generating a rogue-like where
				the behavior of the enemies is generated. This forces the player
				to learn a new set of completely novel enemies each run. The
				specification of chess-like games will help guide us in
				designing a flexible specification to use for generating enemy
				aesthetics and behavior.

			\item \cite{emilio2010}. This paper describes an algorithm for
				playing \textit{Ms. Pac-Man} based on the behavior of ant
				colonies. The paper lists several useful behaviors of ants in
				ant colonies, and how their heuristics can be applied to solving
				game problems such as guiding Ms. Pac-Man through a maze filled
				with hostile ghosts. The researchers used a genetic algorithm to
				optimize the parameters of these ant behaviors, and were able to
				create an effective game-playing AI.

				The researchers' technique for designing agent-based AI inspired
				us to try using a similar method for our enemy AI.
		\end{itemize}
	\item Six external sources we want to read next:
		\begin{itemize}
			\item \cite{jain2016autoencoders}
			\item \cite{van2013procedural}
			\item \cite{shaker2016constructive}
			\item \cite{smith2009rhythm}
			\item \cite{dewsbury2016scalable}
			\item \cite{bizopoulos2014randomly}
		\end{itemize}
	\item Three questions or concerns we have related to the project
		\begin{itemize}
			\item Starting from scratch without using any existing engine makes
				a lot more work for us, such as implementing a graphics manager,
				physics system and collision system.
			\item We're concerned with making sure the level designs and
				generated enemies are both interesting and fun to experience,
				without sacrificing one for the other.
			\item How do we make sure the game is playable without being too
				easy or too hard?
		\end{itemize}
\end{itemize}

\bibliographystyle{ACM-Reference-Format}
\bibliography{references}

\end{document}
